\documentclass{resume}

\usepackage[left=0.75in,top=0.6in,right=0.75in,bottom=0.6in]{geometry}

\name{Ryan Dalby}
\address{ Boulder, CO }
\address{ 385~$\cdot$~313~$\cdot$~2723 \\ dalbyryan3@gmail.com \\
https://dalbyryan3.github.io/}

\def\nameskip{\bigskip}
\def\sectionskip{\medskip}

\begin{document}

  \begin{rSection}{Education}

    \begin{rSubsection}{University of Utah College of Engineering}{May 2022}{Master of Science in Mechanical Engineering, Robotics Track}{Overall GPA: 3.963}
      \item[]
    \end{rSubsection}

    \vspace*{-7.5mm}

    \begin{rSubsection}{University of Utah College of Engineering}{May 2022}{Bachelor of Science in Mechanical Engineering, Minor in Computer Science}{Overall GPA: 3.897}
      \item[]
    \end{rSubsection}

    \vspace*{-7.5mm}

  \end{rSection}
  
  \begin{rSection}{Experience}

    \begin{rSubsection}{Seagate Technology}{June 2022 - August 2022}{Engineer II- Robotics Software}{Longmont, Colorado}
    \item Supported the C\#/.NET software of a robotic manufacturing tool. 
    Used Hierarchical State Machine based software framework, finished backlog tasks, tested changes on hardware, and released software to production.
    \item Reduced unnecessary robotic teach-time by implementing a new recipe management system. Interfaced with various internal database systems with limited documentation by successfully networking across multiple teams. This reduced teach-time by up to 10 times per tool when a new product is introduced.
    \item Introduced modern version control practices to existing robotic software by making the software Git-friendly and implementing Gitflow to improve development speed and encourage collaboration. Added effective, previously non-existent, documentation to assist future developers onboard for robotic tool development.
    \end{rSubsection}

    \begin{rSubsection}{Blyncsy Inc.}{Jan 2022 - May 2022}{Deep Learning Intern}{Salt Lake City, Utah}
    \item Completed graduate-level internship with Blyncsy Inc., working in a small team environment. Created proof-of-concept deep learning-based road crack detection and size estimation pipeline to provide department of transportation customers with a near-real-time geospatial map of road conditions from real-time dashcam imagery. Investigated and utilized state-of-the-art deep learning and computer vision datasets, models, and techniques to overcome having no existing labeled data for the machine learning task. Interacted with external vendors and conducted physical experiments to verify techniques and collect ground-truth data. Detailed results in a publishable report.
    \item Received University of Utah School of Computing Graduate Certificate in Deep Learning by completing requisite coursework and internship focusing on machine learning and computer vision and their applications in industry.
    \end{rSubsection}
  
    \begin{rSubsection}{Seagate Technology}{May 2021 - August 2021}{Summer Intern- Robotics Software Engineer}{Longmont, Colorado}
    \item Implemented new software compatibility for a legacy robotic tool in  C\#/.NET by documenting and porting tool IO, implementing SCARA robot kinematics, and creating software to convert learned robot positions to be compatible with the new software. 
    Extensively documented porting process, tested software on hardware, and described a plan for replacing legacy software. 
    This eliminated the need to maintain two separate codebases for two similar machines.
    \item Supported a robotic tool throughout the summer. Completed high priority backlog items, tested on hardware, and released software to production, communicating with teams around the world.
    \item Investigated creating Cognex Vidi machine learning-based models for defect detection and OCR. Labeled data to train models and documented how to develop models.
    \end{rSubsection}
  
    \begin{rSubsection}{Seagate Technology}{May 2020 - December 2020}{Intern III}{Longmont, Colorado}
    \item Leveraged machine learning to predict robotic tool failures. 
    Explored various models such as support vector machines and recurrent convolutional neural networks implemented using Python and Keras. 
    \item Constructed a data wrangling pipeline to extract trainable features from raw log data. 
    \item Created ML.NET machine learning ONNX model consumption platform.
    \item Expanded team’s machine learning knowledge. Integrated machine software with a machine learning data platform using C\#/.NET.
    \item Reliably communicated with remote team members around the world.
    \end{rSubsection}

    \begin{rSubsection}{Code Corporation}{May 2019 - August 2019}{Mechanical Engineer Intern}{Draper, UT}
    \item Utilized the iterative engineering design process through CAD (SOLIDWORKS) and rapid prototyping to develop an injection-molded bracket to combine multiple parts into a single new product.  
    \item Effectively made various changes to existing products through an engineering change order process to reduce manufacturing costs.  Assisted in conducting product testing to determine possible improvements.  
    \item Employed agile project management in a professional work environment to increase productivity and communication.
    \end{rSubsection}
 
  \end{rSection}

  \begin{rSection}{Technical Strengths}
    \begin{tabular}{ @{} >{\bfseries}l @{\hspace{3ex}} l }
      Programming Languages
      & $\cdot$ Most experienced with C\# and Python. \\
      & $\cdot$ Experienced with C++/C, MATLAB, and Java. \\
      & $\cdot$ Limited experience with JavaScript. \\
      
      & \\

      APIs \& Tools 
      & $\cdot$ Most experienced with .NET, Arduino, \\ 
      & \phantom{$\cdot$} scientific Python libraries (Numpy/Pandas/Matplotlib/Scipy), \\
      & \phantom{$\cdot$} PyTorch, Sci-kit Learn, Sci-kit Image, \\
      & \phantom{$\cdot$} Git, Jupyter, Linux (Bash), Windows, MacOS \\
      & \phantom{$\cdot$} VScode, Visual Studio, and Vim. \\
      & $\cdot$ Experienced with SOLIDWORKS (CSWA certified), \\ 
      & \phantom{$\cdot$} Jira (Scrum), LaTeX, and Docker.\\
      & $\cdot$ Limited experience with the ROS and Three.js. \\
    \end{tabular}
  \end{rSection}
  
  \begin{rSection}{Relevant Coursework}
    \begin{tabular}{ @{} >{\bfseries}l @{\hspace{1ex}} l }
      Graduate Coursework
      & Deep Learning, Machine Learning, Artificial Intelligence, \\
      & Deep Learning Capstone, Image Processing, Virtual Reality,\\
      & Advanced Mechatronics, Robotics, Classical Control, and Robot Control.\\

      & \\

      Undergraduate Coursework
      & Numerical Methods, Mechatronics, Engineering Design, \\
      & Software Practice, and Algorithms and Data Structures. \\
    \end{tabular}
  \end{rSection}

\end{document}
